\documentclass[aspectratio=169]{beamer}
\usepackage[utf8]{inputenc}
\usepackage{hyperref}
\usepackage{amsmath,amsfonts,amsthm,bm}
\usepackage{color}
\usepackage{graphicx} % Allows including images
\usepackage{subcaption}
\usepackage{booktabs} % Allows the use of \toprule, \midrule and \bottomrule in tables
\usepackage{tikz}
%\usepackage{pgfplots}
\usepackage{listings}
\usepackage{courier}
\usepackage[version=4]{mhchem}
\usepackage{array}

\lstset{ %
    basicstyle=\scriptsize\ttfamily, % fonts that are used for the code
    breakatwhitespace=false,         % sets if automatic breaks should only happen at whitespace
%breaklines=true,                 % sets automatic line breaking
%captionpos=b,                    % sets the caption-position to bottom
    commentstyle=\color{gray}\textit,    % comment style
    keepspaces=true,                 % keeps spaces in text, useful for keeping indentation of code (possibly needs columns=flexible)
    keywordstyle=\color{blue},       % keyword style
    language=Python,                 % the language of the code
%otherkeywords={*,...},          % if you want to add more keywords to the set
    rulecolor=\color{black},         % if not set, the frame-color may be changed on line-breaks within not-black text (e.g. comments (green here))
    showspaces=false,                % show spaces everywhere adding particular underscores; it overrides 'showstringspaces'
    showstringspaces=false,          % underline spaces within strings only
    showtabs=false,                  % show tabs within strings adding particular underscores
    stringstyle=\color{red}, % string literal style
    tabsize=4,                       % sets default tabsize to 2 spaces
    columns=fixed                    % Using fixed column width (for e.g. nice alignment)
}

\hypersetup{
    colorlinks=true,
    linkcolor=red,
    filecolor=magenta,
    urlcolor=red,
}

\DeclareMathOperator*{\argmax}{argmax}
\DeclareMathOperator*{\argmin}{argmin}
\let \vec \mathbf

\newcommand{\classname}{NANO266}
\newcommand{\classyear}{Fall 2024}
\mode<presentation> {
    \usetheme{CambridgeUS}
    \setbeamertemplate{footline}[text line]{%
        \parbox{\linewidth}{\vspace*{-8pt}\classname\hfill\classyear\hfill\insertpagenumber}}

    %\setbeamertemplate{footline}[page number]
    \setbeamertemplate{navigation symbols}{}
}


\title[\classname Solving the Schr\"odinger Equation for Periodic Solids]{\classname~- Quantum Mechanical Modeling of Materials and Nanostructures\\Solving the Schr\"odinger Equation for Periodic Solids}

\author{Shyue Ping Ong}
\institute[UCSD]{University of California, San Diego\\
\medskip
}
\date{\classyear} % Date, can be changed to a custom date

\begin{document}


    \begin{frame}
        \titlepage % Print the title page as the first slide
    \end{frame}


    \begin{frame}{The World of Materials}

        \begin{table}[]
            \centering
            \begin{tabular}{p{2cm}|p{3cm}|p{4cm}|p{3cm}}
                & Molecules                                  & Liquids/amorphous/etc.    & Crystals                \\
                \hline
                \hline
                Modelled as & Isolated gas phase                         & Challenging for direct QM & Periodic infinite solid \\
                Basis set   & Localized basis functions, e.g., Gaussians & Simplified models         & Plane waves
            \end{tabular}
        \end{table}
    \end{frame}

\begin{frame}{Definition of a crystal}
A crystal is a time-invariant, \textbf{3D arrangement} of atoms or molecules \textbf{on a lattice}.

\begin{figure}
    \centering
    \includegraphics[width=0.6\linewidth]{lectures/figures/7_crystal.png}
    \caption{Perovskite \ce{SrTiO3}}
\end{figure}

\end{frame}


\begin{frame}{Translational symmetry}
All crystals are characterized by translational symmetry

\begin{equation*}
    \vec{t} = u \vec{a} + v \vec{b} + w \vec{c}, u, v, w \in \mathbb{Z}
\end{equation*}

\begin{figure}
    \centering
    \begin{subfigure}{0.25\textwidth}
        \includegraphics[width=\linewidth]{lectures/figures/7_1D_crystal.png}
    \caption{1D}
    \end{subfigure}
    \begin{subfigure}{0.4\textwidth}
        \includegraphics[width=\linewidth]{lectures/figures/7_2D_crystal.png}
    \caption{2D}
    \end{subfigure}
        \begin{subfigure}{0.25\textwidth}
        \includegraphics[width=\linewidth]{lectures/figures/7_3D_crystal.png}
    \caption{3D}
    \end{subfigure}
\end{figure}

\end{frame} 


\begin{frame}{The 14 3D Bravais Lattices}
\begin{columns}
\column{0.7\textwidth}
\begin{figure}
    \centering
    \includegraphics[width=0.6\linewidth]{lectures/figures/7_bravais_lattices.jpg}
\end{figure}
\column{0.3\textwidth}
a: triclinic (anorthic)\\
m: monoclinic\\
o: orthorhombic\\
t: tetragonal\\
h: hexagonal\\
c: cubic\newline
\newline
P: primitive\\
C: C-centered\\
I: body-centered\\
F: face-centered
\end{columns}

\end{frame} 

\begin{frame}{Unit cells}
Infinite number of unit cells for all 3D lattices. 

Always possible to define primitive unit cells for non-primitive lattices, though the full symmetry may not be retained.

\begin{figure}
    \centering
\begin{subfigure}{0.45\textwidth}
\centering
    \includegraphics[width=0.6\textwidth]{lectures/figures/7_conventional_unit_cell.png}
    \caption{Conventional fcc cell}
\end{subfigure}
\begin{subfigure}{0.45\textwidth}
        \centering
        \includegraphics[width=0.6\textwidth]{lectures/figures/7_primitive_unit_cell.png}
    \caption{Primitive fcc cell}
\end{subfigure}
\end{figure}
\end{frame} 

\begin{frame}{Reciprocal Lattice}

\begin{columns}
\column{0.5\textwidth}
For a lattice given by basis vectors $\vec{a_1}$, $\vec{a_2}$ and $\vec{a_3}$, the reciprocal lattice is given basis vectors $\vec{a_1^*}$, $\vec{a_2^*}$ and $\vec{a_3^*}$ where:

\begin{eqnarray*}
    \vec{a_1^*} & = & 2\pi\frac{\vec{a_2} \times \vec{a_3}}{\vec{a_1}.(\vec{a_2} \times \vec{a_3})}\\
    \vec{a_2^*} & = & 2\pi\frac{\vec{a_3} \times \vec{a_1}}{\vec{a_1}.(\vec{a_2} \times \vec{a_3})}\\
    \vec{a_3^*} & = & 2\pi\frac{\vec{a_1} \times \vec{a_2}}{\vec{a_1}.(\vec{a_2} \times \vec{a_3})}
\end{eqnarray*}
\column{0.5\textwidth}
\begin{equation*}
    \vec{a_i^*}.\vec{a_j} = 2\pi \delta_{ij}
\end{equation*}

Reciprocal translation vectors are given by
\begin{equation*}
    \vec{G} = h\vec{a_1^*} + k\vec{a_2^*} + l\vec{a_3^*}
\end{equation*}
\end{columns} 

\end{frame}

\begin{frame}{External Potential for a Periodic Boundary System}

\begin{figure}
    \centering
    \includegraphics[width=0.8\linewidth]{lectures/figures/7_periodic_potential.png}
    \caption{Simple 1D periodic example}
\end{figure}

\end{frame}

\begin{frame}{Electron in a Periodic Potential}

For an electron in a 1D periodic potential with lattice vector $\vec{a}$, we have

\begin{equation*}
    H = - \frac{1}{2} \nabla^2 + V(r)
\end{equation*}

where $V(r)$ is periodic in $a$,

\begin{equation*}
    V(r) = V(r + ma), \forall m \in \mathbb{Z}
\end{equation*}

For any periodic function, we may express it in terms of a Fourier series:

\begin{equation*}
    V(r) = \sum_{n=-\infty}^{\infty} V_n e^{i\frac{2\pi}{a}nr}
\end{equation*}


\end{frame} 

\begin{frame}{Bloch's Theorem}
For a particle in a periodic potential, eigenstates can be written in the form of a Bloch wave:

\begin{equation*}
\psi_{n,\vec{k}}(\vec{r}) = e^{i\vec{k}.\vec{r}}u_{n,\vec{k}}(\vec{r})
\end{equation*} 

where $u_{n,\vec{k}}(\vec{r})$ has the same periodicity as the crystal and $\vec{k}$ is a vector of real numbers known as the \textbf{crystal wave vector}, $n$ is known as the \textbf{band index}. $e^{i\vec{k}.\vec{r}}$ is a ``plane wave''.\newline
\newline

For any reciprocal lattice vector $\vec{K}$, $\psi_{n,\vec{k}+\vec{K}}(\vec{r}) = \psi_{n,\vec{k}}(\vec{r})$, i.e., we only need to care about $\vec{k}$ in the first Brillouin Zone (BZ).

\end{frame} 


\begin{frame}{First BZ for Some Common Lattices}

\begin{figure}
    \centering
    \begin{subfigure}{0.24\textwidth}
        \includegraphics[width=\linewidth]{lectures/figures/7_Brillouin_Zone_2D.png}
    \caption{2D square and hex}
    \end{subfigure}
    \begin{subfigure}{0.24\textwidth}
        \includegraphics[width=\linewidth]{lectures/figures/7_Brillouin_Zone_BCC.png}
    \caption{BCC}
    \end{subfigure}
    \begin{subfigure}{0.24\textwidth}
        \includegraphics[width=\linewidth]{lectures/figures/7_Brillouin_Zone_FCC.png}
    \caption{FCC}
    \end{subfigure}
    \begin{subfigure}{0.24\textwidth}
        \includegraphics[width=\linewidth]{lectures/figures/7_Brillouin_Zone_Hex.png}
    \caption{Hexagonal}
    \end{subfigure}
\end{figure} 

\end{frame} 

\begin{frame}{Bloch Waves}
\begin{figure}
    \centering
    \includegraphics[width=0.5\linewidth]{lectures/figures/7_Bloch_Waves.png}
\end{figure} 
\end{frame} 

\begin{frame}{Plane Waves as a Basis}

From Bloch's Theorem, 
\begin{equation*}
\psi_{n,\vec{k}}(\vec{r}) = e^{i\vec{k}.\vec{r}}u_{n,\vec{k}}(\vec{r})
\end{equation*} 

Since $u_{n,\vec{k}}(\vec{r})$ has the same periodicity as the lattice, it can be written as a Fourier series of the reciprocal lattice:
\begin{eqnarray*}
u_{n,\vec{k}}(\vec{r}) & = & \sum_{\vec{G}} c^{\vec{G}}_{n,\vec{k}} e^{i\vec{G}.\vec{r}}
\end{eqnarray*} 

where $\vec{G} = h \vec{a_1^*} + k \vec{a_2^*} + l \vec{a_3^*}$ is a translation vector in the reciprocal lattice. Plugging this into the first equation, we have:

\begin{equation*}
\psi_{n,\vec{k}}(\vec{r}) = \sum_{\vec{G}} c^{\vec{G}}_{n,\vec{k}} e^{i(\vec{k}+\vec{G}).\vec{r}}
\end{equation*} 

Note that the summation is an infinite one over all reciprocal space translation vectors $\vec{G}$.

\end{frame} 

\begin{frame}{Truncating Plane Waves}

Plane waves offer a systematic way to improve completeness of our solution based on energy. \newline
\newline
For a free electron in a box, the solution to the Scr\"odinger equation is given by:

\begin{eqnarray*}
\psi(\vec{r}) = e^{i\vec{k}.\vec{r}}, E = \frac{\bar{h}^2}{2m}k^2
\end{eqnarray*} 

Analagously, the solution to to the Scr\"odinger equation for a periodic potential is a linear combination of plane waves with corresponding energies:
\begin{eqnarray*}
E_{\vec{k}+\vec{G}} = \frac{\bar{h}^2}{2m}|\vec{k}+\vec{G}|^2
\end{eqnarray*} 

\end{frame} 

\begin{frame}{Energy cutoff}
Solutions with lower energy are more physically important than solutions with higher energies. So we specify an energy cutoff:
\begin{eqnarray*}
E_{cut} = \frac{\bar{h}^2}{2m}|\vec{G}_{cut}|^2
\end{eqnarray*}

In PWSCF, this is specified using the \href{https://www.quantum-espresso.org/Doc/INPUT_PW.html}{``ecutwfc''} parameter. In the Vienna Ab Initio Simulation Package (VASP), it is called \href{https://www.vasp.at/wiki/index.php/ENCUT}{``ENCUT''}.\newline
\newline
Our wave function is then given by a \textit{finite} sum:

\begin{equation*}
\psi_{n,\vec{k}}(\vec{r}) = \sum_{|\vec{k}+\vec{G}<\vec{G_{cut}}|} c^{\vec{G}}_{n,\vec{k}} e^{i(\vec{k}+\vec{G}).\vec{r}}
\end{equation*} 

\end{frame} 

\begin{frame}{Convergence with respect to energy cutoff}
\begin{columns}
\column{0.5\textwidth}
\begin{figure}
    \centering
    \includegraphics[width=0.8\linewidth]{lectures/figures/7_convergence_energy_cutoff.png}
    \caption{Energy per atom of fcc Cu with lattice constant of 3.64 \AA~using a $12 \times 12 \times 12$ $k$ points as a function of energy cutoff.\cite{shollDensityFunctionalTheory2009}}
\end{figure} 
\column{0.5\textwidth}
\begin{alertblock}{Basis Set Consistency}
The same energy cutoff should be used to ensure basis set consistency if you want to compare energies between calculations, e.g.:

\ce{Cu (s) + Pd (s) -> CuPd(s)}
\end{alertblock}
\end{columns} 

\end{frame} 


\begin{frame}{Pseudopotentials (PSPs)}
\begin{columns}

\column{0.4\textwidth}

\begin{figure}
    \centering
    \includegraphics[width=0.6\linewidth]{lectures/figures/7_Pseudopotentials.png}
    \caption{Comparison of the real (blue) and pseudo (red) wavefunctions. The potentials match above the cutoff radius.}
\end{figure} 

\column{0.6\textwidth}
\begin{equation*}
\psi_{n,\vec{k}}(\vec{r}) = \sum_{\vec{G}} c^{\vec{G}}_{n,\vec{k}} e^{i(\vec{k}+\vec{G}).\vec{r}}
\end{equation*} 
\textbf{Problem: }Tightly bound electrons have wavefunctions that oscillate on very short length scales $\implies$ Need a huge cutoff (and lots of plane waves).\newline
\newline
\textbf{Solution:} Pseudopotentials (PSPs) to represent core electrons with a smoothed density to match various important physical and mathematical properties of true ion core.

\end{columns} 
\end{frame} 

\begin{frame}{Types of Pseudopotentials}

\begin{columns}
\column{0.4\textwidth}
\textbf{Norm-conserving (NC)}: Enforces that inside cut-off radius, the norm of the pseudo-wavefunction is identical to the all-electron wavefunction.\newline
\newline
\textbf{Ultrasoft (US)}: Relax NC condition to reduce basis set size further.\newline
\newline
\textbf{Projector-augmented wave (PAW)}: Avoid some problems with USPP
Generally gives similar results as USPP and all-electron in many instances.\cite{kresseUltrasoftPseudopotentialsProjector1999}
\column{0.6\textwidth}
\begin{figure}
    \centering
        \includegraphics[width=\linewidth]{lectures/figures/7_PSP_comparison.png}
    \caption{Calculated properties of selected crystals using LDA: norm-conserving pseudopotentials (NCPPs), projector augmented waves (PAWs), ultrasoft pseudopotentials (USPPs) and linearized APWs.}
\end{figure} 

\end{columns} 

\end{frame} 

\begin{frame}{Choosing PSPs}
Sometimes, several PPs are available with different number of ``valence'' electrons, i.e., electrons not in the core.\newline
\newline
Choice depends on research problem – if you are studying problems where more (semi-core) electrons are required, choose PSP with more electrons.\newline
\newline
But more electrons $\neq$ better results! (e.g., Rare-earth elements)

\end{frame} 


\begin{frame}{Integrations in $k$ space}
For counting of electrons in bands, total energies, etc., need to sum/intergate over states labeled by $\vec{k}$:
\begin{equation*}
<f> = \frac{V_{cell}}{2\pi} \int_{BZ} f(\vec{k}) d\vec{k} \approx \frac{V_{cell}}{2\pi} \sum_{\vec{k}} f(\vec{k})
\end{equation*} 

\textbf{Born-von Karman boundary condition}: For large but finite crystal of volume $V$ with edges $N_1 \vec{a_1}$, $N_2 \vec{a_2}$, $N_1 \vec{a_3}$:

\begin{equation*}
\psi(\vec{r} + N_1 \vec{a_1}) = \psi(\vec{r} + N_2 \vec{a_2}) = \psi(\vec{r} + N_3 \vec{a_3}) = \psi(\vec{r})
\end{equation*} 

For our Bloch plane waves, the compatible set of $\vec{k}$ is given by:
\begin{equation*}
e^{i\vec{k}.(\vec{r} + N_i \vec{a_i})} = e^{i\vec{k}.\vec{r}}\implies e^{i\vec{k}.N_i \vec{a_i}} = 1 \implies \vec{k} = \frac{m_1}{N_1} \vec{g_1} +  \frac{m_2}{N_2} \vec{g_2} +\frac{m_3}{N_3} \vec{g_3}
\end{equation*} 

\end{frame} 


\begin{frame}{Choosing $\vec{k}$ point grids}

\begin{columns}
\column{0.5\textwidth}
\textbf{Sampling at one point}
\begin{itemize}
    \item Baldereschi point\cite{baldereschiMeanValuePointBrillouin1973}: ``Mean-value point'' dictated by the crystal symmetry.
    \item Gamma ($\Gamma$) point $(0,0,0)$: Commonly used to minimize computational cost for large cells.
\end{itemize}

\textbf{Sampling in a Grid}
\begin{itemize}
    \item Monkhorst-Pack\cite{monkhorstSpecialPointsBrillouinzone1976}: Regular grid mesh.
    \item Grids along high-symmetry lines, e.g., for bandstructure calculations.
\end{itemize}

\column{0.5\textwidth}

\begin{figure}
    \centering
    \begin{subfigure}{\textwidth}
    \centering
        \includegraphics[width=0.55\linewidth]{lectures/figures/7_monkhorst_square.png}
    \caption{MK grid for square lattices}
    \end{subfigure}
    \begin{subfigure}{\textwidth}
        \centering
        \includegraphics[width=0.9\linewidth]{lectures/figures/7_monkhorst_hexagonal.png}
    \caption{MK grid for hexagonal lattices}
    \end{subfigure}
\end{figure} 

\end{columns} 

\end{frame} 


\begin{frame}{Convergence with respect to energy cutoff}
\begin{figure}
    \centering
    \includegraphics[width=0.4\linewidth]{lectures/figures/7_convergence_kpoint.png}
    \caption{Energy per atom of fcc Cu with lattice constant of 3.64 \AA~as a function of $M$ for $M \times M \times M$ $k$-point grid.\cite{shollDensityFunctionalTheory2009}}
\end{figure} 
\end{frame} 


\begin{frame}{Irreducible $k$ points}
Symmetry reduces points that need to be evaluated (irreducible Brillouin Zone).
\begin{figure}
    \centering
    \includegraphics[width=0.6\linewidth]{lectures/figures/7_irr_BZ.png}
    \caption{Grids for integration for a 2d square lattice. The left and center figures are equivalent with one point at the origin, and six inequivalent points in the irreducible BZ shown in grey. Right: A shifted special point grid of the same density but with only three inequivalent points. \cite{shollDensityFunctionalTheory2009}}
\end{figure} 
\begin{alertblock}{$k$ grid vs cell volume}
    The $k$-point grid is inversely related to unit cell volume. E.g., if cell is doubled in one direction, $k$-point grid should be halved in that direction to maintain the same sampling density.
\end{alertblock}

\end{frame} 


\begin{frame}{$k$ points for metals}
\begin{columns}
\column{0.5\textwidth}
BZ in metals are divided into occupied and unoccupied regions by Fermi surface, where the integrated functions change discontinuously from non-zero to zero. $\implies$ Extremely dense $k$-point mesh needed for integration.
\textbf{Algorithmic solutions}
\begin{itemize}
    \item \textbf{Tetrahedron method}: Define a tetrahedra that fill reciprocal space and interpolate. Most widely used is Blochl's version.
    \item \textbf{Smearing} Force the function being integrated to be continuous by ``smearing'' out the discontinuity.
\end{itemize}

\column{0.5\textwidth}
\begin{figure}
    \centering
    \includegraphics[width=0.6\linewidth]{lectures/figures/7_Cu_fermi_surface.png}
    \caption{Fermi-surface of Copper (Cu), the color codes the inverse effective mass of the electrons, large effective masses are represented in red.\cite{weismannSeeingFermiSurface2009}}
\end{figure} 

\end{columns} 
\end{frame} 


    \begin{frame}[allowframebreaks]{Bibliography}
        \bibliographystyle{unsrt}
        \bibliography{refs}
    \end{frame}



    \begin{frame}
        \Huge{\centerline{The End}}
    \end{frame}

\end{document}

